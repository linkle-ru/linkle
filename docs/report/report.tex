\documentclass[a5paper,12pt]{extreport}
\usepackage{cmap}

\usepackage[utf8]{inputenc}
\usepackage[T2A]{fontenc}
\usepackage[russian]{babel}
\usepackage[left=2.5cm, right=1.5cm, vmargin=2.5cm]{geometry}
\linespread{1.25}
\usepackage{lipsum}
\usepackage{indentfirst} % отделять первую строку раздела абзацным отступом
\setlength\parindent{5ex}
\usepackage{fancyhdr}

\pagestyle{fancy}
\fancyhf{}
\rhead{\thepage}
\renewcommand{\headrulewidth}{0pt}

\fancypagestyle{plain}{
\fancyhf{}
\rhead{\thepage}}

\usepackage{titlesec}

\titleformat{\chapter}[block]
{\filcenter}
{\thechapter}
{1em}
{\MakeUppercase}{}

\titlespacing*{\chapter}{0pt}{30pt}{*4}

\titleformat{\section}
{}
{\thesection}
{1ex}{}

\titleformat{\section}[block]
{\hspace{\parindent}}
{\thesection}
{1ex}{}

\titleformat{\subsubsection}[runin]
{}
{\thesubsubsection}
{1ex}{}[.]

\titlespacing*{\subsubsection}{\parindent}{*4}{1ex}

\titlespacing*{\section}{0pt}{*4}{*4}

\titlespacing*{\section}{\parindent}{*4}{*4}

\newcommand\chap[1]{%
\chapter*{#1}%
\addcontentsline{toc}{chapter}{#1}}

\addto{\captionsrussian}{\renewcommand*{\contentsname}{Содержание}}

\usepackage{titletoc}

\dottedcontents{chapter}[1.6em]{}{1.6em}{1pc}

\usepackage[hidelinks]{hyperref} % гиперссылки в содержании

\usepackage[tableposition=top,singlelinecheck=false]{caption}
\usepackage{subcaption}

\DeclareCaptionLabelFormat{gostfigure}{Рисунок #2}
\DeclareCaptionLabelFormat{gosttable}{Таблица #2}
\DeclareCaptionLabelSeparator{gost}{~---~}
\captionsetup{labelsep=gost}
\captionsetup*[figure]{labelformat=gostfigure}
\captionsetup*[table]{labelformat=gosttable}
\renewcommand{\thesubfigure}{\asbuk{subfigure}}

\usepackage{minted}
\usemintedstyle{bw}

\frenchspacing

\usepackage{todonotes}

\begin{document}
    % todo: сделать
    % \thispagestyle{empty}

\begin{center}
    Санкт-Петербургский политехнический университет Петра Великого\\
    Институт Компьютерных Наук и Технологий\\
    \bfseries{Высшая школа программной инженерии}
\end{center}

\vspace{20ex}

\begin{center}
{
\LARGE \textbf{ДИПЛОМНАЯ РАБОТА} \\[3ex]
по дисциплине: «Undefined» \\
по теме: «Undefined» \\[3ex]
Undefined
}
\end{center}

\vspace{40ex}

\noindent Выполнил\\
студент гр.23531/21\hfill
\begin{minipage}{0.7\textwidth}
    \hfill \uline{\hspace{3cm}} \hspace{1.1cm} С.А. Фомин
\end{minipage}

\vspace{3ex}

\noindent Руководитель,\\
преподаватель\hfill
\begin{minipage}{0.7\textwidth}
    \hfill \uline{\hspace{3cm}} \hspace{0.5cm} ?.?. Коликова
\end{minipage}

\vspace{3ex}

\vfill

\begin{center}
    Санкт-Петербург\\
    2019
\end{center}



    \chapter

    \chapter{Введение}
    Сервис для сокращения ссылок время от времени нужен каждому пользователю сети Интернет. Будь то для демонстрации их
    крупным планом на слайдах презентаций, с целью облегчения запоминания или в эстетических целях при SMM
    \todo[inline]{раскрыть что такое SMM и кейсы} - ссылки должны быть максимально короткими и легконабираемыми.
    \section{Общая часть}
    \subsection{Описание предметной области}
    Надо узнать какие разделы нужны и главы и тд.
    \todo[inline]{тест}
    Потом я приведу таблицу с конкурентами, распишу почему мой сервис лучше.
%    \lipsum*[3-7][7-8]
    \listoftodos[Список задач]
\end{document}
